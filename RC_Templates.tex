\documentclass[12pt,compress]{beamer}
%\documentclass[handout,t]{beamer}

\batchmode
% \usepackage{pgfpages}
% \pgfpagesuselayout{4 on 1}[letterpaper,landscape,border shrink=5mm]

\usepackage{amsmath,amssymb,enumerate,epsfig,bbm,calc,color,ifthen,capt-of}

\usepackage[all,pdf]{xy}

% add page number
\expandafter\def\expandafter\insertshorttitle\expandafter{%
	\insertshorttitle\hfill%
	\insertframenumber\,/\,\inserttotalframenumber}

\usetheme{Berlin}
\usecolortheme{umji}

\title{Templates of VC210 RC Slides}
\author{Haotian Fu}
\institute{University of Michigan--Shanghai Jiao Tong University Joint Institute}
\date{\today}
\pgfdeclareimage[height=0.5cm]{umji-logo}{umji.pdf}

\logo{\pgfuseimage{umji-logo}\hspace*{0.5cm}}

\AtBeginSection[]
{
  \begin{frame}<beamer>
    \frametitle{Outline}
    \tableofcontents[currentsection]
  \end{frame}
}
\beamerdefaultoverlayspecification{<+->}

% -----------------------------------------------------------------------------
\begin{document}

% -----------------------------------------------------------------------------

\lecture{RC_\#}

\frame{\titlepage}

\section[Outline]{}
\begin{frame}{Outline}
  \tableofcontents
\end{frame}

% -----------------------------------------------------------------------------

\section{Introduction}
\subsection{How to use this template}
\begin{frame}{How to use this template?}
  \begin{itemize}
    \item This template is created to serve with VC210 \footnote{\tiny{General Chemistry held in University of Michigan--Shanghai Jiao Tong University Joint Institute (UMJI)}} recitation classes (RC) in 2021 FA.
    \item This template is written in \LaTeX\ and compiled by XeLaTeX .
    \item Replace any content as you like!
  \end{itemize}
\end{frame}

% -----------------------------------------------------------------------------

\section{Practical Functions}

\subsection{Blocks}
\begin{frame}{An example of blocks}
    \begin{block}{example}
        This is an example of block.
    \end{block}
    \begin{block}{}
        This is another block.
    \end{block}
\end{frame}

\subsection{Figures and Tables}
\begin{frame}{Examples of figures and tables}
    \begin{figure}
        \centering
        \includegraphics[width=0.5\textwidth]{umji.png}
        \caption{An example of figure}
        \label{fig:demofig-1}
    \end{figure}
    \begin{table}[]
        \centering
        \begin{tabular}{c|c}
            1 & 3 \\ \hline
            2 & 4
        \end{tabular}
        \caption{An example of table}
        \label{tab:demotab-1}
    \end{table}
\end{frame}

\subsection{Graphs}
\begin{frame}{Examples of Graphs}
    \[ \xymatrix{
        A \ar[r] & B \ar@(ur,dr)
    } \]
    \[ \xymatrix{
        A \ar@<.5ex>[r]^{f} & \ar@<.5ex>[l]^{g} B
    } \]
    \[ \xymatrix@ru{
        A \ar[r] & B \ar[d] \\
        C \ar[u] & D \ar[l]
    } \]
    
\end{frame}

%------------------------------------------------------------------------------

\section{Conclusions}

\subsection{Where can I learn more?}
\begin{frame}{Questions and Answers}
    Want to know more?
    \begin{itemize}
        \item Browse \url{http://web.mit.edu/smoot/history.htm}.
        \item Smoot's Legacy \url{http://alum.mit.edu/news/AlumniNews/Archive/smoots_legacy}.
        \item Smoot Salute! \url{http://web.mit.edu/spotlight/smoot-salute}.
    \end{itemize}
\end{frame}

% -----------------------------------------------------------------------------

\end{document}